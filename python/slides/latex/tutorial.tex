\documentclass[hyperref={colorlinks, linkcolor=blue, urlcolor=blue}]{beamer}
\usetheme{MLZ-JCNS}
%\usetheme{MLZ-GEMS}
\usepackage{amssymb}
\usepackage{listings}
%%\usepackage[colorlinks]{hyperref}

%\footinfo{#3}%

\lstset{basicstyle=\ttfamily\normalsize} % for code listings

\lstset{language=Python,
  backgroundcolor=\color{white!97!black},
  keywordstyle=\color{green!40!black},
  identifierstyle=\color{blue!40!black},
  stringstyle=\color{red!60!black},
  commentstyle=\color{green!60!black},
  morecomment=[l][\color{magenta}]{\#},
  morekeywords={print, import, as, from},
  showspaces=false,
  showstringspaces=false,
  tabsize=4,
  deletekeywords={sum},
}


%% \lstset{language=Python,
%%                 keywordstyle=\color{blue},
%%                 stringstyle=\color{red},
%%                 commentstyle=\color{green},
%%                 morecomment=[l][\color{magenta}]{\#}
%% }

%%%%%%%%%%%%%%%%%%%%%%%%%%%%%%%%%%%%%%%%%%%%%%%%%%%%%%%%%%%%%%%%%%%%%%%%%%%%%%%
%%   title                                                                   %%
%%%%%%%%%%%%%%%%%%%%%%%%%%%%%%%%%%%%%%%%%%%%%%%%%%%%%%%%%%%%%%%%%%%%%%%%%%%%%%%

\title{Introduction to Python}
\author{Jonathan Fisher \\ Scientific computing group \\ JCNS at MLZ} 
\date{November  21, 2016}

\begin{document}

%%%%%%%%%%%%%%%%%%%%%%%%%%%%%%%%%%%%%%%%%%%%%%%%%%%%%%%%%%%%%%%%%%%%%%%%%%%%%%%
\part{data formats} 
%%%%%%%%%%%%%%%%%%%%%%%%%%%%%%%%%%%%%%%%%%%%%%%%%%%%%%%%%%%%%%%%%%%%%%%%%%%%%%%

\maketitle



%%%%%%%%%%%%%%%%%%%%%%%%%%%%%%%%%%%%%%%%%%%%%%%%%%%%%%%%%%%%%%%%%%%%%%%%%%%%%%%
\part{Setting up a Python environment} 
%%%%%%%%%%%%%%%%%%%%%%%%%%%%%%%%%%%%%%%%%%%%%%%%%%%%%%%%%%%%%%%%%%%%%%%%%%%%%%%

\begin{frame}
  \frametitle{Useful links}
  %% Slides, example source code, etc. may be obtained at \\
  BornAgain: \\
  \url{http://bornagainproject.org} \\
  \url{https://github.com/scgmlz/BornAgain-tutorial}
  \bigskip

  Python official tutorial: \\ \url{https://docs.python.org/3/tutorial/}
  \bigskip

  Anaconda Python: \\ \url{https://www.continuum.io/}
  \bigskip
  
  PyCharm IDE: \\ \url{https://www.jetbrains.com/pycharm/}
  \bigskip

  \textbf{BornAgain Win/Mac requires Python 2.7}
  \end{frame}

\begin{frame}[fragile]
  \frametitle{Checking the Python Environment}

  Check the Python version:
  \begin{lstlisting}[language=bash]
 $ python --version
 Python 3.5.2
  \end{lstlisting}

  Check for numpy and matplotlib:
  \begin{lstlisting}
 $ python -c "import numpy"
 $ python -c "import matplotlib"
  \end{lstlisting}
  
  Check for BornAgain Python module:
    \begin{lstlisting}
 $ python -c "import bornagain"
    \end{lstlisting}

    
\end{frame}

\begin{frame}[fragile]
  \frametitle{Running Python}

  Direct from command line:

  \begin{lstlisting}[language=bash]
 $ python -c  "print 'hello, world'"
 hello, world
  \end{lstlisting}

  \bigskip

  Run script from command line:
  
  \begin{lstlisting}[language=bash]
 $ echo "print 'hello, world'" > hello.py
 $ python hello.py 
 hello, world
  \end{lstlisting}  
\end{frame}

\begin{frame}[fragile]
  \frametitle{Running Python}

  Default interactive interpreter:

  \begin{lstlisting}[language=Python]
 $ python
 >>> x = 5
 >>> x*x
 25
  \end{lstlisting}

  \bigskip

  IPython interactive interpreter:
    \begin{lstlisting}[language=Python]
 $ ipython
 In [1]: x = 5
 In [2]: x*x
 Out[2]: 25
  \end{lstlisting}
  
\end{frame}


\begin{frame}[fragile]
  \frametitle{Running Python}

  IPython interactive notebook:

  \begin{lstlisting}[language=bash]
 $ ipython notebook
  \end{lstlisting}
  \bigskip
  or Jupyter interactive notebook:
  \begin{lstlisting}[language=bash]
 $ jupyter notebook
  \end{lstlisting}
  \bigskip
  Notebook support is included by default with Anaconda, and can be installed
  as an optional package on most Linux distros.

  \bigskip

  Lastly, can run within PyCharm IDE.
\end{frame}

%%%%%%%%%%%%%%%%%%%%%%%%%%%%%%%%%%%%%%%%%%%%%%%%%%%%%%%%%%%%%%%%%%%%%%%%%%%%%%%
\part{Basic Data Types} 
%%%%%%%%%%%%%%%%%%%%%%%%%%%%%%%%%%%%%%%%%%%%%%%%%%%%%%%%%%%%%%%%%%%%%%%%%%%%%%%

\begin{frame}
  \frametitle{Basic Data Types}

  Python has many data types, e.g.
  \begin{itemize}
  \item numeric: int, float, complex
  \item string
  \item boolean values, i.e. true and false
  \item sequences: list, tuple
  \item dict
  \end{itemize}
\end{frame}

\begin{frame}[fragile]
  \frametitle{Numeric Types}

  \begin{lstlisting}
 x = 5          # int
 x = 10**100    # 2.x: long, 3.x: int
 x = 3.141592   # float
 x = 1.0j       # complex
 math.exp(2)    # float
 cmath.sqrt(-1) # complex
 3 / 2          # 2.x: int, 3.x: float
 2 + 3.4        # float
  \end{lstlisting}
  The numpy module is also available for more serious computation (more later).
\end{frame}

\begin{frame}[fragile]
  \frametitle{Strings}

  \begin{lstlisting}
 x = "hello"    # str
 y = 'world'    # str
 x + ' ' + y    # 'hello world'
 
 # '5 + 6 = 11'
 "{} + {} = {}".format(5, 6, 5+6)
  \end{lstlisting}
\end{frame}

\begin{frame}[fragile]
  \frametitle{List}

  \begin{lstlisting}
 x = [1,2,3]      # create a list
 len(x)           # get the length
 x[1]             # access element
 x[1] = 0         # modify element
 x.append(4)      # append element
 x.extend([5, 6]) # extend list
 x[3:5]           # slice list
  \end{lstlisting}
\end{frame}

\begin{frame}[fragile]
  \frametitle{Tuples}
  Tuples are similar to lists, but are immutable:
    \begin{lstlisting}
 >>> x = (1,2,3)
 >>> x[0] = 4
 Traceback (most recent call last):
   File "<stdin>", line 1, in <module>
 TypeError: 'tuple' object does not 
 support item assignment
   \end{lstlisting}
\end{frame}

  \begin{frame}[fragile]
    \frametitle{List Comprehension}

    Comprehension provides a convenient way to create new lists:
  \begin{lstlisting}
 >>> [ i for i in range(5) ]
 [0, 1, 2, 3, 4]
 >>> [ i**2 for i in range(5) ]
 [0, 1, 4, 9, 16]
 >>> the_list = [5, 2, 6, 1]
 >>> [ i**2 for i in the_list ]
 [25, 4, 36, 1]
  \end{lstlisting}
\end{frame}

\begin{frame}[fragile]
  \frametitle{Boolean Values and Comparisons}

  \begin{lstlisting}
 >>> x = True 
 >>> x and False 
 False
 >>> x or False
 True
 >>> 5 > 6
 False
 >>> 5 == 5
 True
 >>> 15%5 == 0
 True
  \end{lstlisting}
\end{frame}

%%%%%%%%%%%%%%%%%%%%%%%%%%%%%%%%%%%%%%%%%%%%%%%%%%%%%%%%%%%%%%%%%%%%%%%%%%%%%%%
\part{Loops Flow Control} 
%%%%%%%%%%%%%%%%%%%%%%%%%%%%%%%%%%%%%%%%%%%%%%%%%%%%%%%%%%%%%%%%%%%%%%%%%%%%%%%

\begin{frame}
  \frametitle{Loops and Flow Control}  
  \begin{itemize}
  \item for
  \item while
  \item if, elif, else
  \item break, continue
  \end{itemize}
\end{frame}

\begin{frame}[fragile]
  \frametitle{For Loops}
  \begin{lstlisting}[language=Python]
 >>> for i in range(4): print(i**2)
 ... 
 0
 1
 4
 9
 >>> for x in ['hello', 'world']: print x
 ...
 hello
 world
  \end{lstlisting}
\end{frame}

\begin{frame}[fragile]
  \frametitle{While Loops}
  \begin{lstlisting}[language=Python]
 >>> x = 5
 >>> while x > 0: x = x-1; print(x)
 ... 
 4
 3
 2
 1
 0
  \end{lstlisting}
\end{frame}

\begin{frame}[fragile]
  \frametitle{if-elif-else}
  \begin{lstlisting}[language=Python]
 if 5 < 0:
     # never reached
     pass
 elif 5 > 1:
     # reached
     pass
 else:
     # never reached
     pass
  \end{lstlisting}
\end{frame}

\begin{frame}[fragile]
  \frametitle{break and continue}

  \begin{itemize}
  \item continue: skip to next iteration of loop
  \item break: break out of loop entirely
  \end{itemize}
  
  \begin{lstlisting}[language=Python]
 for i in range(100):
     if i%3 == 0:
         continue
     if i%7 == 0:
         break
     print i
  \end{lstlisting}
\end{frame}

\begin{frame}
  \frametitle{FizzBuzz}

  "Write a program that prints the numbers from 1 to 100. But for multiples of three print “Fizz”
  instead of the number and for the multiples of five print “Buzz”. For numbers which are multiples
  of both three and five print “FizzBuzz”."

  \url{http://wiki.c2.com/?FizzBuzzTest}
\end{frame}

\begin{frame}[fragile]
  \frametitle{FizzBuzz Solution}
  \begin{lstlisting}[language=Python]
 for i in range(1, 101):
     multiple_of_3 = (i%3)==0
     multiple_of_5 = (i%5)==0
     if multiple_of_3 and multiple_of_5:
         print 'FizzBuzz'
     elif multiple_of_3:
         print 'Fizz'
     elif multiple_of_5:
         print 'Buzz'
     else:
         print i
  \end{lstlisting}

\end{frame}

%%%%%%%%%%%%%%%%%%%%%%%%%%%%%%%%%%%%%%%%%%%%%%%%%%%%%%%%%%%%%%%%%%%%%%%%%%%%%%%
\part{Functions and Modules} 
%%%%%%%%%%%%%%%%%%%%%%%%%%%%%%%%%%%%%%%%%%%%%%%%%%%%%%%%%%%%%%%%%%%%%%%%%%%%%%%

\begin{frame}[fragile]
  \frametitle{Functions}

  We have already seen examples of functions such as math.exp.
  We can define new functions via def:
  \begin{lstlisting}{language=Python}
 def sqaure(x):
     return x**2

 def is_even(x):
    return x%2 == 0
  \end{lstlisting}
\end{frame}


\begin{frame}[fragile]
  \frametitle{Modules}

  Load external modules (built-in or user-defined) via import:
  \begin{lstlisting}{language=Python}
 import math 
 math.pi
 3.141592653589793 
  \end{lstlisting}
  Rename imported modules with as:
  \begin{lstlisting}{language=Python}
 >>> import math as m
 >>> m.sin(m.pi / 2.0)
 1.0
  \end{lstlisting}
\end{frame}

\begin{frame}[fragile]
  \frametitle{User-defined Modules}
  Implement module in the file ``my\_module.py''
  \begin{lstlisting}
 def square(x):
     return x**2
  \end{lstlisting}
  Use the module:
  \begin{lstlisting}
 >>> import my_module
 >>> my_module.square(5)
 25
  \end{lstlisting}
\end{frame}

%%%%%%%%%%%%%%%%%%%%%%%%%%%%%%%%%%%%%%%%%%%%%%%%%%%%%%%%%%%%%%%%%%%%%%%%%%%%%%%
\part{numpy} 
%%%%%%%%%%%%%%%%%%%%%%%%%%%%%%%%%%%%%%%%%%%%%%%%%%%%%%%%%%%%%%%%%%%%%%%%%%%%%%%

\begin{frame}[fragile]
  \frametitle{numpy}
  numpy is used for arrays of numbers:
  \begin{lstlisting}
 >>> import numpy as np
 >>> x = np.array([1,2,3])
 >>> x.sum()
 6
 >>> x.mean()
 2.0
 >>> x[0] + x[2]
 4
  \end{lstlisting}
\end{frame}

\begin{frame}[fragile]
  \frametitle{numpy}
  multidimensional arrays:
  \begin{lstlisting}
 >>> x = np.array([[1,2],[3,4]])
 >>> x.sum()
 10
 >>> x.sum(0)
 array([4, 6])
 >>> x.sum(1)
 array([3, 7])
  \end{lstlisting}
\end{frame}

\begin{frame}[fragile]
  \frametitle{numpy}
  generating arrays:
  \begin{lstlisting}
 >>> np.zeros(3)
 array([ 0.,  0.,  0.])
 >>> np.ones(4)
 array([ 1.,  1.,  1.,  1.])
 >>> np.random.rand(2)
 array([ 0.81387462,  0.41867461])
 >>> np.linspace(-1, 1, num=4)
 array([-1., -0.33333333, 0.33333333, 1.])
  \end{lstlisting}
\end{frame}

\begin{frame}[fragile]
  \frametitle{operating on numpy arrays}
  \begin{lstlisting}
 >>> x = np.linspace(-math.pi, math.pi, 5)
 >>> np.sin(x)
 array([ 0., -1., 0., 1., 0.])
 >>> np.cos(x)
 array([ -1., 0., 1., 0., -1.])
 >>> np.abs(np.sin(x) + np.cos(x))
 array([ 1., 1., 1., 1., 1.])
  \end{lstlisting}
\end{frame}

\begin{frame}[fragile]
  \frametitle{array manipulation routines}
  numpy.flipud, fliplr, transpose, rot90, flatten, ravel
  \begin{lstlisting}
 >>> ....
  \end{lstlisting}
\end{frame}

%%%%%%%%%%%%%%%%%%%%%%%%%%%%%%%%%%%%%%%%%%%%%%%%%%%%%%%%%%%%%%%%%%%%%%%%%%%%%%%
\part{matplotlib} 
%%%%%%%%%%%%%%%%%%%%%%%%%%%%%%%%%%%%%%%%%%%%%%%%%%%%%%%%%%%%%%%%%%%%%%%%%%%%%%%

\begin{frame}[fragile]
  \frametitle{matplotlib}
  Plotting a function of 1 variable:
  \begin{lstlisting}
 >>> from matplotlib import pyplot as plt
 >>> x = np.linspace(-2, 2)
 >>> y = 2 - 2*x**2 + x**4
 >>> plt.plot(x, y)
 >>> plt.show()
  \end{lstlisting}
\end{frame}

\begin{frame}[fragile]
  \frametitle{matplotlib}
  Plotting a function of 2 variables:
  \begin{lstlisting}
 >>> x = np.linspace(-2, 2, num=200)
 >>> y = np.linspace(-2, 2, num=200)
 >>> xx, yy = np.meshgrid(x, y)
 >>> z = xx + xx**2 - yy
 >>> plt.pcolormesh(x, y, z, shading='gouraud')
 >>> plt.show()
  \end{lstlisting}
\end{frame}

\begin{frame}
  \frametitle{matplotlib}
  For a gallery of plots see
  \url{http://matplotlib.org/gallery.html} 
\end{frame}

\begin{frame}
  \frametitle{Plotting Exercise}
  Consider the function 
  \[ f(x,y) = \mathrm{log}(x+\sqrt{-1}y) \]
  for $-2 \leq x,y \leq 2$.
  Create colormap plots of the magnitude, real, and imaginary parts of $f$.
\end{frame}

\begin{frame}[fragile]
  \frametitle{Exercise Solution}

  \begin{lstlisting}
 >>> x = np.linspace(-2, 2, num=100)
 >>> y = np.linspace(-2, 2, num=100)
 >>> xx, yy = np.meshgrid(x, y)
 >>> z = xx + 1.0j*yy
 >>> logz = np.log(z)
 >>> f_mag = np.abs(logz)
 >>> f_real = np.real(logz)
 >>> f_imag = np.imag(logz)
  \end{lstlisting}
\end{frame}


%%%%%%%%%%%%%%%%%%%%%%%%%%%%%%%%%%%%%%%%%%%%%%%%%%%%%%%%%%%%%%%%%%%%%%%%%%%%%%%
\part{Classes and Inheritance} 
%%%%%%%%%%%%%%%%%%%%%%%%%%%%%%%%%%%%%%%%%%%%%%%%%%%%%%%%%%%%%%%%%%%%%%%%%%%%%%%

\begin{frame}[fragile]
  \frametitle{Defining Classes}
  \begin{lstlisting}
 class SomeClass:
     def __init__(self, x):
        self.x = x

     def doSomething():
        print "doing something"

 >>> obj = SomeClass(5)
 >>> obj.doSomething()
 doing something
 >>> obj.x
 5
  \end{lstlisting}
\end{frame}

\begin{frame}[fragile]
  \frametitle{Inheritance}
    \begin{lstlisting}
 class SomeOtherClass(SomeClass):
     def __init__(self, x, y):
         SomeClass.__init__(self, x)
         self.y = y

     def doNothing():
        print "doing nothing"
    \end{lstlisting}
\end{frame}

\begin{frame}[fragile]
  \frametitle{Inheritance}
    \begin{lstlisting}
 >>> other_obj = SomeOtherClass(5, 6)
 >>> other_obj.doSomething()
 doing something
 >>> other_obj.x
 5
 >>> other_obj.y
 6
 >>> other_obj.doNothing()
 doing nothing 
  \end{lstlisting}
\end{frame}

\begin{frame}[fragile]
  \frametitle{Polymorphism}

  \begin{lstlisting}
  >>> isinstance(obj, SomeClass)
  True
  >>> isinstance(obj, SomeOtherClass)
  False
  >>> isinstance(other_obj, SomeClass)
  True
  >>> isinstance(other_obj, SomeOtherClass)
  True
  \end{lstlisting}
\end{frame}

\begin{frame}
  \frametitle{Exercise}
\end{frame}


%%%%%%%%%%%%%%%%%%%%%%%%%%%%%%%%%%%%%%%%%%%%%%%%%%%%%%%%%%%%%%%%%%%%%%%%%%%%%%%
\part{Bonus Material} 
%%%%%%%%%%%%%%%%%%%%%%%%%%%%%%%%%%%%%%%%%%%%%%%%%%%%%%%%%%%%%%%%%%%%%%%%%%%%%%%

\begin{frame}
  \frametitle{Bonus Material}

  These remaining slides contain additional topics that we probably
  won't have time to cover during the tutorial.
\end{frame}

\begin{frame}
  \frametitle{ctypes}
\end{frame}

\begin{frame}
  \frametitle{Python 2 vs. 3}
  Key differences include:
  \begin{itemize}
  \item print statement
  \item integer division
  \item int vs. long
  \item new style classes
  \item some standard modules/functions have been moved/renamed
  \end{itemize}
  The ``\_\_future\_\_'' module can be used to write code compatible with both Python 2 and 3.
\end{frame}

\begin{frame}
  \frametitle{git and github}
\end{frame}

\begin{frame}[fragile]
  \frametitle{Building BornAgain with Python 3 support}
    \begin{lstlisting}[language=bash]
 $ git clone \
 > https://github.com/scgmlz/BornAgain.git
 $ mkdir build; cd build
 $ cmake .. \
 > -DCMAKE_BUILD_TYPE=Release \
 > -DBORNAGAIN_USE_PYTHON3=ON
 $ make && make install
    \end{lstlisting}



\end{frame}


%%%%%%%%%%%%%%%%%%%%%%%%%%%%%%%%%%%%%%%%%%%%%%%%%%%%%%%%%%%%%%%%%%%%%%%%%%%%%%%
\end{document}
%%%%%%%%%%%%%%%%%%%%%%%%%%%%%%%%%%%%%%%%%%%%%%%%%%%%%%%%%%%%%%%%%%%%%%%%%%%%%%%
