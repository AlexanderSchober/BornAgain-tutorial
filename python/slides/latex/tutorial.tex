\documentclass[hyperref={colorlinks, linkcolor=blue, urlcolor=blue}]{beamer}
\usetheme{MLZ-JCNS}
%\usetheme{MLZ-GEMS}
\usepackage{amssymb}
\usepackage{listings}
%%\usepackage[colorlinks]{hyperref}

%\footinfo{#3}%

\lstset{basicstyle=\ttfamily\normalsize} % for code listings

\lstset{language=Python,
  backgroundcolor=\color{white!97!black},
  keywordstyle=\color{green!40!black},
  identifierstyle=\color{blue!40!black},
  stringstyle=\color{red!60!black},
  commentstyle=\color{green!60!black},
  morecomment=[l][\color{magenta}]{\#},
  morekeywords={print, import, as, from},
  showspaces=false,
  showstringspaces=false,
  tabsize=4,
  deletekeywords={sum},
}


%% \lstset{language=Python,
%%                 keywordstyle=\color{blue},
%%                 stringstyle=\color{red},
%%                 commentstyle=\color{green},
%%                 morecomment=[l][\color{magenta}]{\#}
%% }

%%%%%%%%%%%%%%%%%%%%%%%%%%%%%%%%%%%%%%%%%%%%%%%%%%%%%%%%%%%%%%%%%%%%%%%%%%%%%%%
%%   title                                                                   %%
%%%%%%%%%%%%%%%%%%%%%%%%%%%%%%%%%%%%%%%%%%%%%%%%%%%%%%%%%%%%%%%%%%%%%%%%%%%%%%%

\title{Introduction to Python}
\author{Jonathan Fisher \\ Scientific computing group \\ JCNS at MLZ} 
\date{November  21, 2016}

\begin{document}

%%%%%%%%%%%%%%%%%%%%%%%%%%%%%%%%%%%%%%%%%%%%%%%%%%%%%%%%%%%%%%%%%%%%%%%%%%%%%%%
\part{data formats} 
%%%%%%%%%%%%%%%%%%%%%%%%%%%%%%%%%%%%%%%%%%%%%%%%%%%%%%%%%%%%%%%%%%%%%%%%%%%%%%%

\maketitle



%%%%%%%%%%%%%%%%%%%%%%%%%%%%%%%%%%%%%%%%%%%%%%%%%%%%%%%%%%%%%%%%%%%%%%%%%%%%%%%
\part{Setting up a Python environment} 
%%%%%%%%%%%%%%%%%%%%%%%%%%%%%%%%%%%%%%%%%%%%%%%%%%%%%%%%%%%%%%%%%%%%%%%%%%%%%%%

\begin{frame}
  \frametitle{Useful links}
  %% Slides, example source code, etc. may be obtained at \\
  BornAgain: \\
  \url{http://bornagainproject.org} \\
  \url{https://github.com/scgmlz/BornAgain-tutorial}
  \bigskip

  Python official tutorial: \\ \url{https://docs.python.org/3/tutorial/}
  \bigskip

  Anaconda Python: \\ \url{https://www.continuum.io/}
  \bigskip
  
  PyCharm IDE: \\ \url{https://www.jetbrains.com/pycharm/}
  \bigskip

  \textbf{BornAgain Win/Mac requires Python 2.7}
  \end{frame}

\begin{frame}[fragile]
  \frametitle{Checking the Python Environment}

  Check the Python version:
  \begin{lstlisting}[language=bash]
 $ python --version
 Python 3.5.2
  \end{lstlisting}

  Check for numpy and matplotlib:
  \begin{lstlisting}
 $ python -c "import numpy"
 $ python -c "import matplotlib"
  \end{lstlisting}
  
  Check for BornAgain Python module:
    \begin{lstlisting}
 $ python -c "import bornagain"
    \end{lstlisting}

    
\end{frame}

\begin{frame}[fragile]
  \frametitle{Running Python}

  Direct from command line:

  \begin{lstlisting}[language=bash]
 $ python -c  "print 'hello, world'"
 hello, world
  \end{lstlisting}

  \bigskip

  Run script from command line:
  
  \begin{lstlisting}[language=bash]
 $ echo "print 'hello, world'" > hello.py
 $ python hello.py 
 hello, world
  \end{lstlisting}  
\end{frame}

\begin{frame}[fragile]
  \frametitle{Running Python}

  Default interactive interpreter:

  \begin{lstlisting}[language=Python]
 $ python
 >>> x = 5
 >>> x*x
 25
  \end{lstlisting}

  \bigskip

  IPython interactive interpreter:
    \begin{lstlisting}[language=Python]
 $ ipython
 In [1]: x = 5
 In [2]: x*x
 Out[2]: 25
  \end{lstlisting}
  
\end{frame}


\begin{frame}[fragile]
  \frametitle{Running Python}

  IPython interactive notebook:

  \begin{lstlisting}[language=bash]
 $ ipython notebook
  \end{lstlisting}
  \bigskip
  or Jupyter interactive notebook:
  \begin{lstlisting}[language=bash]
 $ jupyter notebook
  \end{lstlisting}
  \bigskip
  Notebook support is included by default with Anaconda, and can be installed
  as an optional package on most Linux distros.

  \bigskip

  Lastly, can run within PyCharm IDE.
\end{frame}

%%%%%%%%%%%%%%%%%%%%%%%%%%%%%%%%%%%%%%%%%%%%%%%%%%%%%%%%%%%%%%%%%%%%%%%%%%%%%%%
\part{Basic Data Types} 
%%%%%%%%%%%%%%%%%%%%%%%%%%%%%%%%%%%%%%%%%%%%%%%%%%%%%%%%%%%%%%%%%%%%%%%%%%%%%%%

\begin{frame}
  \frametitle{Basic Data Types}
\end{frame}

%%%%%%%%%%%%%%%%%%%%%%%%%%%%%%%%%%%%%%%%%%%%%%%%%%%%%%%%%%%%%%%%%%%%%%%%%%%%%%%
\part{Flow Control} 
%%%%%%%%%%%%%%%%%%%%%%%%%%%%%%%%%%%%%%%%%%%%%%%%%%%%%%%%%%%%%%%%%%%%%%%%%%%%%%%

\begin{frame}
  \frametitle{Flow Control}
\end{frame}

\begin{frame}
  \frametitle{FizzBuzz}

  "Write a program that prints the numbers from 1 to 100. But for multiples of three print “Fizz”
  instead of the number and for the multiples of five print “Buzz”. For numbers which are multiples
  of both three and five print “FizzBuzz”."

  \url{http://wiki.c2.com/?FizzBuzzTest}
\end{frame}

\begin{frame}[fragile]
  \frametitle{FizzBuzz Solution}
  \begin{lstlisting}[language=Python]
 for i in range(1, 101):
     multiple_of_3 = (i%3)==0
     multiple_of_5 = (i%5)==0
     if multiple_of_3 and multiple_of_5:
         print 'FizzBuzz'
     elif multiple_of_3:
         print 'Fizz'
     elif multiple_of_5:
         print 'Buzz'
     else:
         print i
  \end{lstlisting}

\end{frame}


%%%%%%%%%%%%%%%%%%%%%%%%%%%%%%%%%%%%%%%%%%%%%%%%%%%%%%%%%%%%%%%%%%%%%%%%%%%%%%%
\part{Modules} 
%%%%%%%%%%%%%%%%%%%%%%%%%%%%%%%%%%%%%%%%%%%%%%%%%%%%%%%%%%%%%%%%%%%%%%%%%%%%%%%

\begin{frame}[fragile]
  \frametitle{Modules}

  Load external modules (built-in or user-defined) via import:
  \begin{lstlisting}{language=Python}
 import math 
 math.pi
 3.141592653589793 
  \end{lstlisting}
  Rename imported modules with as:
  \begin{lstlisting}{language=Python}
 >>> import math as m
 >>> m.sin(m.pi / 2.0)
 1.0
  \end{lstlisting}
\end{frame}

%%%%%%%%%%%%%%%%%%%%%%%%%%%%%%%%%%%%%%%%%%%%%%%%%%%%%%%%%%%%%%%%%%%%%%%%%%%%%%%
\part{numpy} 
%%%%%%%%%%%%%%%%%%%%%%%%%%%%%%%%%%%%%%%%%%%%%%%%%%%%%%%%%%%%%%%%%%%%%%%%%%%%%%%

\begin{frame}[fragile]
  \frametitle{numpy}
  numpy is used for arrays of numbers:
  \begin{lstlisting}
 >>> import numpy as np
 >>> x = np.array([1,2,3])
 >>> x.sum()
 6
 >>> x.mean()
 2.0
 >>> x[0] + x[2]
 4
  \end{lstlisting}
\end{frame}

\begin{frame}[fragile]
  \frametitle{numpy}
  multidimensional arrays:
  \begin{lstlisting}
 >>> x = np.array([[1,2],[3,4]])
 >>> x.sum()
 10
 >>> x.sum(0)
 array([4, 6])
 >>> x.sum(1)
 array([3, 7])
  \end{lstlisting}
\end{frame}

\begin{frame}[fragile]
  \frametitle{numpy}
  generating arrays:
  \begin{lstlisting}
 >>> np.zeros(3)
 array([ 0.,  0.,  0.])
 >>> np.ones(4)
 array([ 1.,  1.,  1.,  1.])
 >>> np.random.rand(2)
 array([ 0.81387462,  0.41867461])
 >>> np.linspace(-1, 1, num=4)
 array([-1., -0.33333333, 0.33333333, 1.])
  \end{lstlisting}
\end{frame}

\begin{frame}[fragile]
  \frametitle{operating on numpy arrays}
  \begin{lstlisting}
 >>> x = np.linspace(-math.pi, math.pi, 5)
 >>> np.sin(x)
 array([ 0., -1., 0., 1., 0.])
 >>> np.cos(x)
 array([ -1., 0., 1., 0., -1.])
 >>> np.abs(np.sin(x) + np.cos(x))
 array([ 1., 1., 1., 1., 1.])
  \end{lstlisting}
\end{frame}

\begin{frame}[fragile]
  \frametitle{array manipulation routines}
  numpy.flipud, fliplr, transpose, rot90, flatten, ravel
  \begin{lstlisting}
 >>> ....
  \end{lstlisting}
\end{frame}

%%%%%%%%%%%%%%%%%%%%%%%%%%%%%%%%%%%%%%%%%%%%%%%%%%%%%%%%%%%%%%%%%%%%%%%%%%%%%%%
\part{matplotlib} 
%%%%%%%%%%%%%%%%%%%%%%%%%%%%%%%%%%%%%%%%%%%%%%%%%%%%%%%%%%%%%%%%%%%%%%%%%%%%%%%

\begin{frame}[fragile]
  \frametitle{matplotlib}
  Plotting a function of 1 variable:
  \begin{lstlisting}
 >>> from matplotlib import pyplot as plt
 >>> x = np.linspace(-2, 2)
 >>> y = 2 - 2*x**2 + x**4
 >>> plt.plot(x, y)
 >>> plt.show()
  \end{lstlisting}
\end{frame}

\begin{frame}[fragile]
  \frametitle{matplotlib}
  Plotting a function of 2 variables:
  \begin{lstlisting}
 >>> x = np.linspace(-2, 2, num=200)
 >>> y = np.linspace(-2, 2, num=200)
 >>> xx, yy = np.meshgrid(x, y)
 >>> z = xx + xx**2 - yy
 >>> plt.pcolormesh(x, y, z, shading='gouraud')
 >>> plt.show()
  \end{lstlisting}
\end{frame}

\begin{frame}
  \frametitle{matplotlib}
  For a gallery of plots see
  \url{http://matplotlib.org/gallery.html} 
\end{frame}

\begin{frame}
  \frametitle{Plotting Exercise}
  Consider the function 
  \[ f(x,y) = \mathrm{log}(x+\sqrt{-1}y)| \]
  for $-2 \leq x,y \leq 2$.
  Create colormap plots of the magnitue, real, and imaginary parts of $f$.
\end{frame}

\begin{frame}[fragile]
  \frametitle{Exercise Solution}

  \begin{lstlisting}
 >>> x = np.linspace(-2, 2, num=100)
 >>> y = np.linspace(-2, 2, num=100)
 >>> z = np.array([ [ u-1.0j*v for u in x ] for v in y])
 >>> logz = np.log(z)
 >>> f_mag = np.abs(logz)
 >>> f_real = np.real(logz)
 >>> f_imag = np.imag(logz)
  \end{lstlisting}
\end{frame}


%%%%%%%%%%%%%%%%%%%%%%%%%%%%%%%%%%%%%%%%%%%%%%%%%%%%%%%%%%%%%%%%%%%%%%%%%%%%%%%
\part{Classes} 
%%%%%%%%%%%%%%%%%%%%%%%%%%%%%%%%%%%%%%%%%%%%%%%%%%%%%%%%%%%%%%%%%%%%%%%%%%%%%%%

\begin{frame}
  \frametitle{Classes}
\end{frame}

\begin{frame}
  \frametitle{Inheritance}
\end{frame}

\begin{frame}
  \frametitle{Polymorphism}
\end{frame}

\begin{frame}
  \frametitle{Exercise}
\end{frame}


%%%%%%%%%%%%%%%%%%%%%%%%%%%%%%%%%%%%%%%%%%%%%%%%%%%%%%%%%%%%%%%%%%%%%%%%%%%%%%%
\part{Bonus Material} 
%%%%%%%%%%%%%%%%%%%%%%%%%%%%%%%%%%%%%%%%%%%%%%%%%%%%%%%%%%%%%%%%%%%%%%%%%%%%%%%

\begin{frame}
  \frametitle{ctypes}
\end{frame}

\begin{frame}
  \frametitle{Python 2 vs. 3}
\end{frame}

\begin{frame}
  \frametitle{git and github}
\end{frame}

\begin{frame}[fragile]
  \frametitle{Building BornAgain with Python 3 support}
    \begin{lstlisting}[language=bash]
 $ git clone \
 > https://github.com/scgmlz/BornAgain.git
 $ mkdir build; cd build
 $ cmake .. \
 > -DCMAKE_BUILD_TYPE=Release \
 > -DBORNAGAIN_USE_PYTHON3=ON
 $ make && make install
    \end{lstlisting}



\end{frame}


%%%%%%%%%%%%%%%%%%%%%%%%%%%%%%%%%%%%%%%%%%%%%%%%%%%%%%%%%%%%%%%%%%%%%%%%%%%%%%%
\end{document}
%%%%%%%%%%%%%%%%%%%%%%%%%%%%%%%%%%%%%%%%%%%%%%%%%%%%%%%%%%%%%%%%%%%%%%%%%%%%%%%
